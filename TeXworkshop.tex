\documentclass[a4paper, 11pt,fleqn]{jarticle}               % fleqn:  数式左寄せ

\usepackage[utf8]{inputenc}        % rtf 経由で word に出力する時に必要,  latex2rtfで日本語
\usepackage{booktabs}                    % 表に罫線を引くパッケージ
\usepackage{lscape}                              % 横向きのページを作るパッケージ
\usepackage{color}                              % 文字の色を変えるパッケージ
\usepackage{okumacro}                    % ルビをふるパッケージ
\usepackage{times}                         %  times: LaTex で Times New Roman
\usepackage{txfonts}                         % 数式で使うフォントの指定
\usepackage[margin=25mm,bottom=20mm,left=25mm, right=25mm]{geometry}          %  [LaTeX] 1ページ当たりの行数,文字数と周囲の余白を設定する


% ハイフネーションの禁止 -----
\hyphenpenalty=10000\relax
\exhyphenpenalty=10000\relax
\sloppy


% 図表の挿入とキャプションの設定 -----
\usepackage{wrapfig}
\usepackage[dvipdfm]{graphicx}
\usepackage{mediabb}
\renewcommand{\figurename}{Fig.}          % Latexで図表番号だけを英語で表記,  キャプションのスタイルを変更したい スタイルファイル
\renewcommand{\tablename}{Table }
\usepackage{here}          % 図を確実に入れ込むためのパッケージ here or float


% 引用に関する設定 -----
\usepackage{natbib}
\bibliographystyle{jecon}
\renewcommand{\refname}{引用文献}


% しおりとハイパーリンクを作るパッケージ -----
\AtBeginDvi{\special{pdf:tounicode 90ms-RKSJ-UCS2}}
\usepackage[dvipdfm,
                      bookmarks=true,
                      bookmarksnumbered=true,
                      bookmarkstype=toc, 
                      colorlinks=true,
                      linkcolor=black, 
                      citecolor=blue,
                      urlcolor=black,
                      pdfborder={0 0 0}]{hyperref}




% DRAFT マーク
%\usepackage{graphicx}
%\usepackage{color}
\usepackage{fancybox}
\fancyput(2.0cm, -18.0cm){
 \color[rgb]{0.85,0.85,0.85}{\rotatebox{50}{\scalebox{10}{DRAFT}}}}


% 行間の設定
\usepackage{setspace} % setspaceパッケージのインクルード
\setstretch{1.5} % ページ全体の行間を設定



% 関数の設定 //////////////////////////////////////////////////////

% 追加箇所 -----
\usepackage{color}
\definecolor{green}{rgb}{0, 0.6, 0}	% 緑色を再定義
\newcommand{\NS}[1]{\textcolor{green}{#1}}


% 共著者へのメッセージや自分へのメモ -----
\newcommand{\memo}[1]{{\footnotesize \textcolor{red}{[#1]}}}



% 当落線上の一文 -----
\newcommand{\reduce}[1]{\textcolor{magenta}{#1}}




%//////////////////////////////////////////////////////////////////////////////////////////////////////////////////////////////////////////////////
\begin{document}                    % ここから文章開始

\title{\TeX 講習会}
\date{2018 年 4 月}
\author{中野 善、 林 晃}
\maketitle

\tableofcontents

\newpage
\section{はじめに}

本講習会では、
\begin{itemize}
\item \TeX による事業報告書の編集ができるようになること
\item \TeX による論文の執筆に対する抵抗感を下げること
\end{itemize}
を目的としています。

本講習を受講される皆様におかれましては、事業報告書の執筆および編集作業、学術論文の執筆や研究費申請書の作成等、あらゆる種類の執筆を日夜行っていることと思います。
一般に編集・執筆作業では、そのエフォートのほとんどは「論理展開の構成」と「体裁の統一」に割かれます。
事前のご準備をお願いします。
\begin{itemize}
\item TeX
\item SumatraPDF
\item Notepad++
\item R
\item Inkscape
\end{itemize}

% -------------------------------------------------------------------------------------------
\subsection*{TeX}
\TeX の開発環境の構築には、Windows では \TeX インストーラーを用い、Mac では \memo{...林くん、お願いします...} を用います。
詳しくは以下のホームページを御覧ください。


\noindent Windows

\begin{itemize}
\item 簡単\LaTeX\footnote{\TeX にマクロを実装したもの。ユーザーは \LaTeX を通して \TeX を使うことになる。本講習では両者を区別しない。}インストールWindows編\footnote{\url{http://did2memo.net/2016/04/24/easy-latex-install-windows-10-2016-04/}}
\item \TeX インストーラー\footnote{\url{http://www.math.sci.hokudai.ac.jp/~abenori/soft/abtexinst.html}}
\end{itemize}



\noindent Mac

\memo{...林くん、お願いします...}



% -------------------------------------------------------------------------------------------
\subsection*{SumatraPDF}
SumatraPDF は Adobe Acrobat Reader DC 等と同じ PDF ビューワーです。
一般に PDF を作成・更新するときは、ファイルを閉じておかなければいけません。
しかし、この SumatraPDF はファイルを開いた状態でPDF を作成・更新できます。
つまり、出力結果を確認しながら編集作業を行えるという優れものであり、\TeX Wiki\footnote{\url{https://texwiki.texjp.org/?SumatraPDF}} でも推奨されています。
インストールの際は \href{https://www.sumatrapdfreader.org/download-free-pdf-viewer.html}{SumatraPDF インストーラー}\footnote{\url{https://www.sumatrapdfreader.org/download-free-pdf-viewer.html}} をご利用下さい。
% -------------------------------------------------------------------------------------------
\subsection*{R}
R は統計解析向けのプログラミング言語である R 言語を扱う開発環境です。
様々なパッケージが開発されており、これを実装することで複雑な統計解析も簡単なコードで実行できるようになります。
今回は \TeX で表を作成する際にパッケージ Hmisc を使用しますので、予めインストールして下さい。
R をインストールする際は \href{https://cran.ism.ac.jp/bin/windows/}{R for Windows}\footnote{\url{https://cran.ism.ac.jp/bin/windows/}} または \href{https://cran.ism.ac.jp/bin/macosx/}{R for Mac OS X}\footnote{\url{https://cran.ism.ac.jp/bin/macosx/}} よりインストーラーをご利用下さい。
パッケージ Hmisc をインストールする際は、R のコンソールに install.packages("Hmisc") と入力して下さい。





% -------------------------------------------------------------------------------------------
\subsection*{Inkscape}
Inkscape は Adobe Illustrator のようなドロー系の描画ソフトです。
本講習では R で作成したグラフの整形の際に少し触れます。
受講にあたってインストールする必要はありませんが、インストールの際は \href{https://inkscape.org/en/release/0.92.2/}{Inkscape インストーラー}\footnote{\url{https://inkscape.org/en/release/0.92.2/}} をご利用下さい。
























\end{document}                    % ここで文章終り
