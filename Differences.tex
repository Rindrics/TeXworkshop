\documentclass[TeXworkshop]{subfiles}
\begin{document}
\clearpage

\section{\TeX のすゝめ}
\subsection{ワープロソフトの種類}
動作に着目すると、文書を作成するソフトウェアは大きく2タイプに分類することができます。
1つは、テキストの編集画面をほぼそのまま出力するWYSIWY\kenten{G} [What You See Is What You \kenten{G}et]タイプ、
% フリガナ追加
もう1つは、文書の構造を明示的に定義して編集を行うWYSIWY\kenten{M} [What You See Is What You \kenten{M}ean]タイプです。これら2タイプのソフトウェアは、得意とする作業が大きく異なっています。


私たちは、日々の研究活動や業務において様々な種類の文書を扱いますが、その特性の大部分は文書の規模、内容の反復度、数値データの数、相互参照や図表の数などによって決定されます。
品質要求を満たした文書を最短時間で作成するためには、
自身が作成しようとしている文書の特性を見極め、適切なタイプのソフトウェアを使用する必要があります。

それでは、これら2つのタイプの文書作成ソフトウェアは、どのように異なるのでしょうか。

\subsubsection{WYSIWYG}
WYSIWYGタイプのソフトウェアの中にも、シンプルなものから高機能なものまで様々ありますが、ある程度の規模の文書を作成する場合には、Microsoft Wordのように、図表の挿入から校正機能など、多くの機能を備えた「オール・インワン」的なソフトウェアを使用することが多いようです(表 \ref{table:differences})。これらはマウスを用いた直感的な操作性に優れているため、使用開始初期の学習コストが低いのが特長です。


\subsubsection{WYSIWYM}
WYSIWYMタイプのソフトウェアでは、編集画面が見たままに出力されないため、文書内の図表などの要素の位置をテキストで細かく指定したり、文章にさまざまな属性を与えることができます。
本講習で紹介する\TeX は、このWYSIWYMタイプのソフトウェアの代表格です。
\TeX では、目次や図表、索引、脚注、数式番号、文献などの参照も、ファイル名や参照のための手がかりを埋め込んでおくことによって、容易に実現できます。
「○○ページを御覧ください」のような、特定ページの参照も可能です。
この方式をとっていることによって、\TeX で作成された文書は変更に強く、再現性を高く保つことができます(表 \ref{table:differences})。 
文書の体裁に注意力を削がれずに済むため、執筆者は残りの作業時間の大部分を文書内容の推敲に充てることができます。

\TeX は、変更を成果物に反映させるためにコンパイルを必要とします。
この作業によって、文書ファイル内に埋め込まれた体裁に関する命令や、相互参照の手がかりを拾い集め、最終的な成果物を作成します。
コンパイルは一見、煩わしい手間に思えるかもしれませんが、\TeX 側ではユーザーからの入力を常時監視せずに済むため、大規模な文書を扱う際にも動作を軽快に保つことができます。

\begin{table}[h]
  \begin{center}
  \caption{Microsoft Wordと\TeX の長所の違い}
  \label{table:differences}
  \begin{tabular}{rcc}
                & Word & \TeX \\\hline
    機能の数              & $\circ$ & $\times$\\
    直感的操作            & $\circ$ & $\times$\\
    学習コスト            & $\circ$ & $\times$\\
    再現性                & $\times$ & $\circ$ \\
    相互参照              & $\times$ & $\circ$ \\
    大規模文書の扱いやすさ& $\times$ & $\circ$ \\\hline
  \end{tabular}
  \end{center}
\end{table}

\begin{itembox}[l]{メモ}
\TeX の隠れた特長に、仕上がりの美しさが挙げられます。
  これは、\TeX が単なるワープロソフトではなく、美しい文書を作成することを目的として開発された組版ソフトであるためです。
  \TeX の開発者であるDonald Knuth博士は、自著をあたかも組版職人によって組版されたかのように美しく仕上げたいと考え、1976年に\TeX の開発を始めました。
\end{itembox}

\subsection{報告書・論文作成には\TeX を}
WYSIWIG/WYSIWYM両タイプのソフトウェア(以下Microsoft Wordと\TeX)は場面に応じて適切に使い分けることが重要です。
このことを理解するために、万能包丁と蕎麦切り包丁のアナロジーを考えてみましょう。
料理(文書)の特性に合わせて、包丁(ソフトウェア)を使い分けるのは自然なことです。

今あなたが、ちょっと小腹が空いてしまったとします。
自分で食べるちょっとした一皿を、冷蔵庫にあるあり合わせの材料で、手早く美味しく作りたい、という状況で、蕎麦切り包丁を手にする人はいないでしょう。
こんな時には万能包丁の出番です。

では、あなたがお蕎麦屋さんの蕎麦打ち職人だったらどうでしょうか。
お客様にあなたの自慢の蕎麦を美味しく食べてもらうには、少なくとも麺は\kenten{均一}な太さでないといけません。
ここで使うべきは間違いなく蕎麦切り包丁で、あなたは店の個性を麺の味わいやコシ、太さなどで表現するはずです。

もうお気づきとは思いますが、「ちょっとした一皿と万能包丁」はちょっとした覚え書きとMicrosoft Wordの、「自慢の蕎麦と蕎麦切り包丁」は報告書や学術論文と\TeX のアナロジーです。毎年、一定のフォーマットで提出する報告書、IMRAD型式の学術論文は、\TeX で作成しましょう。

\subsection{\TeX ファイルの構造}
\TeX で文書を作成するための.texファイルは、文書の構造や仕様を定義するための「プリアンブル」、そして実際に印刷される文書の内容は「ドキュメント」の2部に分かれています。
それぞれの役割を見ていきましょう。

\subsubsection{プリアンブル}
プリアンブルは、.texファイルの先頭に位置しています。
\subsubsection{ドキュメント}
プリアンブルの下、\ttfamily{\begin{verbatim}\begin{document}\end{verbatim}}
と\ttfamily{\begin{verbatim}\end{document}\end{verbatim}}
に挟まれた部分は「ドキュメント」と呼ばれ、実際の執筆作業のほとんどはこの部分で行います。
\subsection{関数の紹介}

\end{document}

