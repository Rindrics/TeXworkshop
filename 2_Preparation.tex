\documentclass[TeXworkshop]{subfiles}
\begin{document}
\section{環境設定 ---講習会の前に---}

% -------------------------------------------------------------------------------------------
\subsection*{\TeX のインストール}
Windows、Mac共に、\TeX 執筆環境を構築するための便利なインストーラ、パッケージが準備されています。
ご自分のOSに合わせて、以下を参考にご準備ください。

\subsubsection*{Windows}

\begin{itemize}
\item 簡単\LaTeX\footnote{\TeX にマクロを実装したもの。ユーザーは \LaTeX を通して \TeX を使うことになる。本講習では両者を区別しない。}インストールWindows編\footnote{\url{http://did2memo.net/2016/04/24/easy-latex-install-windows-10-2016-04/}}
\item \TeX インストーラ\footnote{\url{http://www.math.sci.hokudai.ac.jp/~abenori/soft/abtexinst.html}}
\end{itemize}

\subsubsection*{Mac}
\TeX の執筆環境を一括整備してくれるMac\TeX パッケージをご利用いただくと便利です。
公式サイト\footnote{\url{http://www.tug.org/mactex/mactex-download.html}}より\ttfamily{mactex-20******.pkg}をダウンロードして実行し、画面の指示に従ってインストールを進めてください。
インストールが完了したら、試しにターミナル.appを開き、\ttfamily{tex}と入力してみましょう。\TeX が起動し、"\ttfamily{This is TeX, Version 3.xx (TeX Live 2017) (preloaded format=tex)
**}"などと応答が返って来れば準備完了です(動作確認後、ターミナルは閉じて構いません)。\ttfamily{Unknown command 'tex'}というエラーが返ってきた場合にはインストールに失敗しております。主催者までご連絡下さい。

\begin{itembox}[l]{メモ}
Mac\TeX とは、以下のソフトウェアをバンドルした統合パッケージです。
\begin{itemize}
	\item{\TeX Shop}(執筆、コンパイル、プレビューソフト)
	\item{\TeX Live Utility}(パッケージ管理ソフト)
	\item{BibDesk}(文献管理ソフト)
	\item{\LaTeX iT}(数式作成用ソフト)
	\item{Excalibur}(スペルチェックソフト)
\end{itemize}
\end{itembox}




\subsubsection*{その他のOS}
あなたは自力でできるはずです。
講習には、いつもお使いの言語の開発環境、ドロー系描画ソフト、PDFビューアを準備しておいて下さい。

% -------------------------------------------------------------------------------------------
\subsection*{その他のソフトウェア}
本講習では、実際の研究業務の作業フローを紹介するため、以下のソフトウェアを使用します。
事前にご準備をお願いします(同様の機能を持つソフトウェアを既にお使いの方は、準備の必要はありません)。
\begin{itemize}
\item R(プログラミング言語)
\item Notepad++(テキストエディタ)
\item Inkscape(ベクター編集ソフト)
\item SumatraPDF(PDFビューア)
\end{itemize}
\subsection*{SumatraPDF}
SumatraPDF は Adobe Acrobat Reader DC 等と同じ PDF ビューワーです。
一般に PDF を作成・更新するときは、ファイルを閉じておかなければいけません。
しかし、この SumatraPDF はファイルを開いた状態でPDF を作成・更新できます。
つまり、出力結果を確認しながら編集作業を行えるという優れものであり、\TeX Wiki\footnote{\url{https://texwiki.texjp.org/?SumatraPDF}} でも推奨されています。
インストールの際は \href{https://www.sumatrapdfreader.org/download-free-pdf-viewer.html}{SumatraPDF インストーラー}\footnote{\url{https://www.sumatrapdfreader.org/download-free-pdf-viewer.html}} をご利用下さい。





% -------------------------------------------------------------------------------------------
\subsection*{Notepad++}
Notepad++ は Windows に標準装備されているメモ帳と同じようなテキストエディターです。
今回使用する \TeX に加え、R, Python, C, C++ 等の主要なプログラムをシンタックスハイライト\footnote{シンタックスハイライト (英: syntax highlighting) とは、テキストエディタの機能であり、テキスト中の一部分をその分類ごとに異なる色やフォントで表示するものである (Wikipedia より)。}で表示し、入力をサポートしてくれるソフトウェアです。
インストールの際は \href{https://notepad-plus-plus.org/download/v7.5.3.html}{Notepad++ インストーラー}\footnote{\url{https://notepad-plus-plus.org/download/v7.5.3.html}} をご利用下さい。






% -------------------------------------------------------------------------------------------
\subsection*{R}
R は統計解析向けのプログラミング言語である R 言語を扱う開発環境です。
様々なパッケージが開発されており、これを実装することで複雑な統計解析も簡単なコードで実行できるようになります。
今回は \TeX で表を作成する際にパッケージ Hmisc を使用しますので、予めインストールして下さい。
R をインストールする際は \href{https://cran.ism.ac.jp/bin/windows/}{R for Windows}\footnote{\url{https://cran.ism.ac.jp/bin/windows/}} または \href{https://cran.ism.ac.jp/bin/macosx/}{R for Mac OS X}\footnote{\url{https://cran.ism.ac.jp/bin/macosx/}} よりインストーラーをご利用下さい。
パッケージ Hmisc をインストールする際は、R のコンソールに install.packages("Hmisc") と入力して下さい。





% -------------------------------------------------------------------------------------------
\subsection*{Inkscape}
Inkscape は Adobe Illustrator のようなドロー系の描画ソフトです。
本講習では R で作成したグラフの整形の際に少し触れます。
受講にあたってインストールする必要はありませんが、インストールの際は \href{https://inkscape.org/en/release/0.92.2/}{Inkscape インストーラー}\footnote{\url{https://inkscape.org/en/release/0.92.2/}} をご利用下さい。


\end{document}















