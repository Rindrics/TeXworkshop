\clearpage

\section{報告書編集作業の実習 1}
\subsection{最小限の \TeX ファイルを作る}
はじめに WinShell を起動して、以下のコードを入力し、Ctrl + S で適当なフォルダに保存します。
このとき、ファイル名を「報告書編集作業の実習1」とでもしておきましょう。



% ソースコード --------------------------------------------------------------------------
\begin{lstlisting}[basicstyle=\gt\ttfamily\footnotesize, frame=single]
\documentclass[a4paper, 11pt]{jarticle}

[プリアンブル部]

% ////////////////////////////////////////////
\begin{document}		% ここから文章開始

[本文]

\end{document}			% ここで文章終わり
\end{lstlisting}
% -----------------------------------------------------------------------------------------



\noindent
つぎに [本文] に適当な文字を入力、コンパイルできるか確認します。
例えば以下のように入力します。


\vspace{\Cvs}
\noindent
◆◆◆ 入力 ◆◆◆
% ソースコード --------------------------------------------------------------------------
\begin{lstlisting}[basicstyle=\gt\ttfamily\footnotesize, frame=single]
\documentclass[a4paper, 11pt]{jarticle}

% ////////////////////////////////////////////
\begin{document}		% ここから文章開始

はじめての \TeX

\end{document}			% ここで文章終わり
\end{lstlisting}
% -----------------------------------------------------------------------------------------


\noindent
コンパイルは WinShell で F5 を押し、PDF 作成は F10 を押すと実行されます。
その後、F11 を押して作成した PDF を開くと以下のように記述されていることが確認できます。


\vspace{\Cvs}
\noindent
◆◆◆ 出力 ◆◆◆
\begin{screen}
はじめての \TeX
\end{screen}

\noindent
このように、[本文] に文字を入力し、PDF が作成されることが確認できました。
つぎは、この [本文] に報告書の文章を入力していきます。




\subsection{報告書の \TeX ファイルを作る}
報告書編集作業用として、{\gt\ttfamily 編集用wordファイル\_相関係数.docx} と {\gt\ttfamily 編集用wordファイル\_ヒストグラム.docx} を準備しました。
これは Wikipedia より作成したものです。
これ以降、これらの Word ファイルを \TeX で編集・統合する作業を説明します。

基本的には報告書の本文を Word から Winshell の [本文] にコピペし、\TeX コードを追加していく作業になります。
\TeX コードとは以下で説明する節や数式などのことです。

報告書本文をコピペする際は少量ずつ行うことをお薦めします。
なぜなら、全文コピペを行うとコンパイルエラーが出たときに検討する項目が増え、作業の進捗が遅れるからです。
少量ずつコピペし、確実にコンパイルできることを確認しながら着実に作業を進めるほうが良いでしょう。




\subsubsection{改行する}
\TeX の改行には次のようなルールがあります。
	\begin{itemize}
	\item 1 回の改行だけでは段落は変わらない
	\item 段落は空の行\footnote{全角スペースが入っていると空の行と認識されないが、半角スペースはいくら入っていても空の行とみなされる。}で作成される
	\item 空の行は何行あっても 1 行とみなされる
	\end{itemize}

手始めに {\gt\ttfamily 編集用wordファイル\_相関係数.docx} のタイトルである「相関係数」から「1 概要」の最後までを Word から Winshell の [本文] にコピペします。


\vspace{\Cvs}
\noindent
◆◆◆ 入力 ◆◆◆
% ソースコード --------------------------------------------------------------------------
\begin{lstlisting}[basicstyle=\gt\ttfamily\footnotesize, frame=single]
\documentclass[a4paper, 11pt]{jarticle}

% ////////////////////////////////////////////
\begin{document}		% ここから文章開始

相関係数
著者名:Wikipedia


1	概要
相関係数(そうかんけいすう、英: correlation coefficient)は、2つの確率変数の間にある線形な関係の強弱を測る指標である。相関係数は無次元量で、-1以上1以下の実数に値をとる。相関係数が正のとき確率変数には正の相関が、負のとき確率変数には負の相関があるという。また相関係数が0のとき確率変数は無相関であるという。
たとえば、先進諸国の失業率と実質経済成長率は強い負の相関関係にあり、相関係数を求めれば比較的-1に近い数字になる。
相関係数が+1または-1に値をとるのは2つの確率変数が線形な関係にあるとき、かつそのときに限る。また2つの確率変数が互いに独立ならば相関係数は0となるが、逆は成り立たない。
普通、単に相関係数といえばピアソンの積率相関係数を指す。ピアソン積率相関係数の検定は偏差の正規分布を仮定する(パラメトリック)方法であるが、他にこのような仮定を置かないノンパラメトリックな方法として、スピアマンの順位相関係数、ケンドールの順位相関係数なども一般に用いられる。


\end{document}			% ここで文章終わり
\end{lstlisting}
% -----------------------------------------------------------------------------------------



\vspace{\Cvs}
\noindent
◆◆◆ 出力 ◆◆◆◆◆◆◆◆◆◆◆◆◆◆◆◆◆◆◆◆◆

相関係数
著者名:Wikipedia


1	概要
相関係数(そうかんけいすう、英: correlation coefficient)は、2つの確率変数の間にある線形な関係の強弱を測る指標である。相関係数は無次元量で、-1以上1以下の実数に値をとる。相関係数が正のとき確率変数には正の相関が、負のとき確率変数には負の相関があるという。また相関係数が0のとき確率変数は無相関であるという。
たとえば、先進諸国の失業率と実質経済成長率は強い負の相関関係にあり、相関係数を求めれば比較的-1に近い数字になる。
相関係数が+1または-1に値をとるのは2つの確率変数が線形な関係にあるとき、かつそのときに限る。また2つの確率変数が互いに独立ならば相関係数は0となるが、逆は成り立たない。
普通、単に相関係数といえばピアソンの積率相関係数を指す。ピアソン積率相関係数の検定は偏差の正規分布を仮定する(パラメトリック)方法であるが、他にこのような仮定を置かないノンパラメトリックな方法として、スピアマンの順位相関係数、ケンドールの順位相関係数なども一般に用いられる。

\vspace{\Cvs}
\noindent
◆◆◆◆◆◆◆◆◆◆◆◆◆◆◆◆◆◆◆◆◆◆◆◆◆◆









\subsubsection{節・小節(項)・小々節(目)を作る}


\subsubsection{図を入れる}


\subsubsection{数式を書く}


\subsubsection{箇条書きする}