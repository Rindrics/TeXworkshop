\documentclass{jsarticle}
\usepackage[dvipdfmx, hiresbb]{graphicx, xcolor}
\title{平成\ThisYrJP(\ThisYr)年度カタクチイワシ対馬暖流系群の資源評価}
\author{}
\date{}

\newcommand{\ThisYr}{2018}
\newcommand{\ThisYrJP}{30}

\newenvironment{TwoFigs}[6]{
\begin{figure}[h]
 \begin{minipage}{0.5\hsize}
  \begin{center}
   \includegraphics[width=70mm]{#1}
  \end{center}
  \caption{#2}
  \label{#3}
 \end{minipage}
 \begin{minipage}{0.5\hsize}
  \begin{center}
   \includegraphics[width=70mm]{#4}
`  \end{center}
  \caption{#5}
  \label{#6}
 \end{minipage}
\end{figure}}

\begin{document}
\maketitle
\section{まえがき}
\section{生態}
\section{漁業の状況}
\section{資源の状態}
\section{\ThisYr 年ABCの算定}
\section{ABC以外の管理方策の提言}

\clearpage
\TwoFigs
{figs/age_length.png}{カタクチイワシ対馬暖流系群の分布域}{fig:distrib}
{figs/age_length.png}{カタクチイワシの成長様式}{fig:agelen}

\TwoFigs
{figs/maturation.png}{年齢別成熟率}{fig:age_matur}
{figs/catch.png}{カタクチイワシとシラスの漁獲量}{fig:catch}

\TwoFigs
{figs/eggp.png}{産卵量の経年変化}{fig:eggp}
{figs/index_cpue.png}{現存量指標値}{fig:index_cpue}

\end{document}
